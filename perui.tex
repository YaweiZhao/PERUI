\documentclass[letterpaper]{article}
\usepackage{aaai}
\usepackage{times}
\usepackage{helvet}
\usepackage{courier}
%%%%%%%%%%%%%%%%%%%%%%%%%%%%%%%%%%%%%%%%%%%%%%%%%

\usepackage{graphicx}
\usepackage{subfigure}
\usepackage{epstopdf}
\usepackage{mdwmath}
\usepackage{mdwtab}
\usepackage{amsthm}
\usepackage{amssymb}
\usepackage{amsmath}
\usepackage{algorithm}
%\usepackage{algorithmic}
\usepackage[noend]{algpseudocode}
\usepackage{booktabs}
\usepackage{array}
\usepackage{multirow}
\usepackage{hyperref}
\usepackage{balance}
\usepackage{bm}
\usepackage{ulem}

\usepackage{pifont}
\usepackage{epsfig,tabularx}
%\usepackage{mathptmx}
\usepackage{url}
\usepackage[marginal]{footmisc}

\frenchspacing
\setlength{\pdfpagewidth}{8.5in}
\setlength{\pdfpageheight}{11in}
\pdfinfo{
/Title (PERUI: A General Framework of Reduced Variance Stochastic Gradient Gradient and the Hybrid Implementation)
/Author (Yawei Zhao, Yuewei Ming, Jianping Yin)}
\setcounter{secnumdepth}{0}
 \begin{document}
% The file aaai.sty is the style file for AAAI Press
% proceedings, working notes, and technical reports.
%
\title{PERUI: A General Framework of Reduced Variance Stochastic Gradient Gradient and the Hybrid Implementation}
\author{Yawei Zhao\\
School of Computer\\
National University\\
of Defense Technology\\
Changsha, China, 410073\\
\And
Yuewei Ming\\
School of Computer\\
National University\\
of Defense Technology\\
Changsha, China, 410073\\
\And
Jianping Yin\\
School of Computer\\
National University\\
of Defense Technology\\
Changsha, China, 410073\\
}
%\author{Paper ID: 112}
\maketitle
\begin{abstract}

\end{abstract}

\section{Introduction}
\label{sect_introduction}



\section{Related work}
\label{sect_related_work}




\section{The general hybrid framework of SGD}
\label{sect_framework}


\begin{algorithm}[t]
    \caption{The general framework of variance reduced SGD: PERUI}
    \label{algorithm_combine}
    \begin{algorithmic}[1]
        \Require $\omega^0\in \mathbb{R}^d$. $\forall i\in[n]$, and $[n]$ represents $\{1,2, ..., n\}$.
        \State \textbf{\uline{P}robability:} $[i_t]\leftarrow \mathcal{P}([n])$ where $i_t \in \{1,2, ..., n\}$. $t$ is a positive integer;
        \State \textbf{\uline{E}poch:} the sequence of the epoch size $\{m^0, m^1, ..., m^S\}\leftarrow \mathcal{E}([i_t])$;
        \For {$s=0,1,2,...,S$}
            \State $\omega_0^s=\tilde{\omega}^s$;
            \State $g=\frac{1}{n}\sum\limits{i=1}^n\nabla f_i(\tilde{\omega}^s)$;
            \For {$t<m^s$}
                \State \textbf{\uline{R}educed variance:} $v=\mathcal{R}(\nabla f_{i_t}(\omega_{i_t}^t)-\nabla f_{i_t}(\tilde{\omega}^s))$;
                \State $\gamma_t^s=v+g$;
                \State \textbf{\uline{U}pdate:} $\omega_{t+1}^s=\mathcal{U}(\eta_t, \omega_t^s, \gamma_t^s)$;
            \EndFor
            \textbf{\uline{I}dentification:} $\tilde{w}^{s+1}\leftarrow\mathcal{I}([\omega_j^s])$ with $j\in\{1,2, ..., m^s\}$;
        \EndFor
        \Return $\tilde{\omega}^S$.
    \end{algorithmic}
\end{algorithm}




%\begin{figure}
%\centering
%\subfigure{\includegraphics[width=0.86\columnwidth]{figures/evaluation_accelerated_factor}}
%\caption{The learning rate with the acceleration factor makes DisSVRG converge fast significantly.}
%\label{figure_evaluation_accelerated_factor}
%\end{figure}





\section{Convergence analysis}
\label{sect_convergence_analysis}








\section{Optimization}
\label{sect_optimization}



\subsection{Constant learning rate with an acceleration factor}
\label{sect_acceleration_factor}







\subsection{Adaptive update sharing strategy}
\label{sect_update_sharing}




\section{Discussion}
\label{sect_discussion}









\section{Performance evaluation}
\label{sect_performance_evaluation}




\subsection{Convergence}
\label{sect_performance_evaluation_convergence}







\subsection{Speed up}
\label{sect_performance_evaluation_speed_up}







\subsection{Wait time}
\label{sect_performance_evaluation_wait_time}






\subsection{Parallel threads}
\label{sect_performance_evaluation_parallel_threads}





\section{Conclusion}
\label{sect_conclusion}








\section*{Acknowledgements}

\bibliography{reference}
\bibliographystyle{aaai}

\end{document}
